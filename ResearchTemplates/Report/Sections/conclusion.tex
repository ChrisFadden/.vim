\section{Conclusion} \label{sec:conclusion}
The technique described is inherently a frequency-domain method.  Therefore, in 
order to calculate the reflection and transmission over a wide bandwidth, 
multiple simulations are necessary.  The Scatter Matrix Method, like its 
analytical analogue the Transfer Matrix Method, is most used in the optics field.  
The Scatter Matrix Method has applications involving lasers incident on 
crystalline structures, as well as fiber-optic communication, and optical 
resonators.  Any optical application dealing with a wave incident upon a layered 
structure is the problem this technique was designed to solve.  As well as optics, 
this concept can also be applied to the field of acoustics.  Acousticians derived 
an acoustic wave equation, using pressure and particle velocity, which is closely 
related to the derivation given for the electromagnetic fields.  Therefore, with 
only slight modification, the Scattering Matrix Method could be used in 
underwater acoustics, modelling the environmental impact of a noise source, or 
the efficiency of noise control on a room or concert hall.  This method is not 
well suited for extremely accurate radio frequency or microwave applications.  
The 1-D approximation is not as valid at these frequencies when high accuracy is 
required, and therefore a more rigorous technique such as the FDTD should be used.  

The Scatter Matrix Method derivation described is not the only way to model 
structures using scattering matrices.  The techniques of Rigorous Coupled Wave 
Analysis and Method of Lines are closely related to this technique.  
As well as using different techniques, a more 
efficient calculation could involve parallelizing the code involved.  Each layer 
involved is completely independent from the others, and does not interact until 
the collapse into the total scattering matrix at the end.  Therefore, this method 
should almost trivially parallelize.  Computational parallelization such as 
threads or MPI, as well as GPU processing using CUDA or OpenCL would make 
calculating extremely complicated structures much more feasible. 


