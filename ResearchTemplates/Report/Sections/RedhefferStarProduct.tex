\section{Redheffer Star Product}\label{sec:redStar}
The analytical version of the Scatter Matrix Method uses transfer matrices.  
When you combine different transfer matrices in a material, it is simply matrix 
multiplication.  However, the transfer matrix method is computationally unstable.  
This is because it treats all waves as forward propagating, and backwards-decaying 
waves are treated as growing exponentials.  In order 
to remedy this, it is computationally stable to use scatter matrices, which 
treat waves as both forward and backward propagating.  When combining scatter 
matrices however, the relationship is not simple matrix multiplication, but the 
Redheffer Star Product.  

Two adjacent scattering matrices will have the following form, taking A to be 
$C_1$ from Fig. 3 %\ref{fig:ScatterDiagram}
, and B to be $C_2$: 

  \begin{equation}
 \left[ {\begin{array}{c}
               A_1^- \\
               B_1^+ \\
               \end{array} } \right]
            = \left[ {\begin{array}{cc}
               S_{11}^1 & S_{12}^1 \\
               S_{21}^1 & S_{22}^1 \\
               \end{array} } \right]
            \left[ {\begin{array}{c}
               A_1^+ \\
               B_1^- \\
               \end{array} } \right]
    \label{eq:17}
    \end{equation}
  \begin{equation}
 \left[ {\begin{array}{c}
               B_1^- \\
               A_2^+ \\
               \end{array} } \right]
            = \left[ {\begin{array}{cc}
               S_{11}^2 & S_{12}^2 \\
               S_{21}^2 & S_{22}^2 \\
               \end{array} } \right]
            \left[ {\begin{array}{c}
               B_1^+ \\
               A_2^- \\
               \end{array} } \right]
    \label{eq:18}
    \end{equation}

 Through matrix algebra, there is a way to relate $A_1^-$ and $A_2^+$ to $A_1^+$ 
 and $A_2^-$.  This relationship is the combined scattering matrix for the two 
 layers.  The formulas for the individual components of the total scattering 
 matrix is given below:

  \begin{equation}
  S_{11}^{tot} = S_{11}^1 + S_{12}^1[I - S_{11}^2S_{22}^1]^{-1} S_{11}^2S_{21}^1
  \label{eq:19}
  \end{equation}
  \begin{equation}
  S_{12}^{tot} = S_{12}^1[I - S_{11}^2S_{22}^1]^{-1} S_{12}^2
  \label{eq:20}
  \end{equation}
  \begin{equation}
  S_{21}^{tot} = S_{21}^2[I - S_{22}^1S_{11}^2]^{-1} S_{21}^1
  \label{eq:21}
  \end{equation}
  \begin{equation}
  S_{22}^{tot} = S_{22}^2 + S_{21}^2[I - S_{22}^1S_{11}^2]^{-1} S_{22}^1S_{12}^2
  \label{eq:22}
  \end{equation}

%\begin{figure}[here]
%\centering
%\includegraphics[width=2.5in]{TransmissionReflectionMatrices}
%\caption{Transmission and Reflection S-Matrix}
%\label{fig:Smatrix}
%\end{figure}

 The boundary layers of the structure are categorized as an incident reflection 
 layer, and a final transmission layer.  Therefore, the boundary scattering 
 matrices require a slightly different form, compactly given in Fig 4. %\ref{fig:Smatrix}.  
 In order to find the total scatter matrix for the entire structure, all layers 
 must be combined using the Redheffer Star Product: 

 \begin{equation}
  S^{tot} = S_{ref} \otimes S_1 \otimes S_2 ... \otimes S_{trans}
  \label{eq:23}
  \end{equation}


