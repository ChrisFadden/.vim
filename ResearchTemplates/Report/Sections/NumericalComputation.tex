\section{Numerical Computation} \label{sec:NumComp}
The real benefit of this technique is that, while it may not be feasible to do 
this by hand for extended structures, each step can be done using a computer.  
The simulation starts by determining the source parameters.  To reduce floating 
point error, the wavelength is set to 1.0, and the size of the structure is given 
in terms of multiples of a wavelength.  The size of the structure, as well its 
electrical characteristics permittivity and permeability, can all be collected 
in separate arrays.  The simulation will have a loop through the structure, 
indexing these arrays for each layer.  

 The eigenvalue problem at the heart of this technique can actually be partially 
 solved before the simulation begins.  The matrices in equations (\ref{eq:8}) and 
 (\ref{eq:9}) are diagonal.  The eigenvectors of a diagonal matrix are the 
 standard basis vectors, meaning W can be represented as the identity matrix.  
 The eigenvalue is $–k_z^2$, with the square root being $jk_z$, the propagation 
 constant.

  Since the eigenvalue problem can be solved before the simulation, the loop can 
  efficiently use the solutions to create the A, B, and X helper matrices for 
  each layer.  To solve for the scattering matrix from the helper matrices is 
  simply matrix algebra, which the computer will have no problems doing.  The 
  final step in the loop is combining the layer matrix with the total scattering 
  matrix, through the use of the Redheffer Star Product.  The star product could 
  be a simple function call, taking in two scattering matrices and performing the 
  product, again simple matrix algebra.  While computers will be limited by the 
  range of numbers floating point numbers can represent, the only general 
  approximation is simplifying the problem to 1D.  Therefore, numerical 
  artifacts should not create a significant error from analytical solutions.


