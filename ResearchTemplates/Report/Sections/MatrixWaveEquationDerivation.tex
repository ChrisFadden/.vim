\section{Matrix Wave Equation Derivation}\label{sec:wave}
The coordinate system that will be used in the derivation is the Cartesian 
coordinate system.  Therefore, instead of the more common r, $\theta$, and 
$\phi$ coordinates, the coordinates will be given in terms of x, y, and z.  
Since this method is commonly used in the optical community, and the fact that 
there are no boundaries except in one direction, standing wave modes are just 
described by their general direction.  They do not play a larger role in the 
derivation, the equations describing their direction are given here for 
completeness:
\begin{equation}
a_{TE} = \frac{k_{inc} \times n}{|| {k_{inc} \times n ||} }
\label{eq:1}
\end{equation}
\begin{equation}
a_{TM} = \frac{ a_{TE} \times k_{inc} } {|| a_{TE} \times k_{inc} || }
\label{eq:2}
\end{equation}

 While Maxwell’s Equations totally govern electrodynamics, the two equations 
 that represent coupled traveling waves are the curl equations.  For numerical 
 stability, this method uses a "normalized" magnetic field, which is scaled by 
 the impedance.  With the normalized field, and in phasor form, the curl 
 equations are given as: 

 \begin{equation}
 \nabla \times E = k_0 \mu_r H
 \label{eq:3}
 \end{equation}
 \begin{equation}
 \nabla \times H = k_0 \epsilon_r E
 \label{eq:4}
 \end{equation}

  The structures analyzed using this technique are assumed to be homogeneous in 
  the x and y directions and have propagation governed by complex exponentials.  
  Therefore, any partial derivatives in those two directions, in general is 
  described by: 

  \begin{equation}
  \frac{\partial A}{\partial w} = j k_w A
  \label{eq:5}
  \end{equation}

   This simplifies the partial derivatives in equations (\ref{eq:3}) and 
   (\ref{eq:4}), and allows any partial derivatives in the z direction to 
   become ordinary derivatives.  These simplifications, and combining like terms 
   gives the following two matrix equations: 

   \small

   \begin{equation}
   \frac{d}{dz} \left[ {\begin{array}{c}
               E_x \\
               E_y \\
               \end{array} } \right]
            = \frac{1}{\epsilon_r}
            \left[ {\begin{array}{cc}
               k_x k_y & \mu_r \epsilon_r - k_x^2 \\
               k_y^2 - \mu_r \epsilon_r & - k_x k_y \\
               \end{array} } \right]
            \left[ {\begin{array}{c}
               H_x \\
               H_y \\
               \end{array} } \right]
    \label{eq:6}
    \end{equation}
   \begin{equation}
   \frac{d}{dz} \left[ {\begin{array}{c}
               H_x \\
               H_y \\
               \end{array} } \right]
            = \frac{1}{\mu_r}
            \left[ {\begin{array}{cc}
               k_x k_y & \mu_r \epsilon_r - k_x^2 \\
               k_y^2 - \mu_r \epsilon_r & - k_x k_y \\
               \end{array} } \right]
            \left[ {\begin{array}{c}
               E_x \\
               E_y \\
               \end{array} } \right]
    \label{eq:7}
    \end{equation}

    \normalsize

     Equations (\ref{eq:6}) and (\ref{eq:7}) give a relationship between the E 
     and H fields, but do not describe propagation in the current form.  The 
     wave equation, which does govern propagation, is given using the second 
     derivative of the given fields.  Substituting equation (\ref{eq:6}) into 
     (\ref{eq:7}), and (\ref{eq:7}) into (\ref{eq:6}), gives the final form of 
     the wave equations:

   \begin{equation}
   \frac{d^2}{d^2z} \left[ {\begin{array}{c}
               E_x \\
               E_y \\
               \end{array} } \right]
            = \frac{1}{\epsilon_r}
            \left[ {\begin{array}{cc}
               -k_z^2 & 0 \\
               0 & -k_z^2 \\
               \end{array} } \right]
            \left[ {\begin{array}{c}
               E_x \\
               E_y \\
               \end{array} } \right]
    \label{eq:8}
    \end{equation}
   \begin{equation}
   \frac{d^2}{d^2z} \left[ {\begin{array}{c}
               H_x \\
               H_y \\
               \end{array} } \right]
            = \frac{1}{\mu_r}
            \left[ {\begin{array}{cc}
               -k_z^2 & 0 \\
               0 & -k_z^2 \\
               \end{array} } \right]
            \left[ {\begin{array}{c}
               H_x \\
               H_y \\
               \end{array} } \right]
    \label{eq:9}
    \end{equation}

 Equations (\ref{eq:8}) and (\ref{eq:9}) are both eigenvalue problems with 
 similar solutions.  The square root of the eigenvalue is the propagation 
 constant through the material.  The eigenvectors are split up to create useful 
 scaling factors.  The A vector found in the solutions is used to create the 
 scattering matrix: 

 \begin{equation}
 \left[ {\begin{array}{c}
               E_x \\
               E_y \\
               \end{array} } \right]
            =  We^{-j \beta z}A^+ + W^{+j \beta z}A^-
 \label{eq:10}
 \end{equation}
  \begin{equation}
 \left[ {\begin{array}{c}
              H_x \\
              H_y \\
              \end{array} } \right]
            = Ve^{-j \beta z}A^+ + V^{-j \beta z}A^-
 \label{eq:11}
 \end{equation}


