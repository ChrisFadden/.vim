\section{Scatter Matrix Derivation}\label{sec:scatter}

%\begin{figure}[here]
%\centering
%\includegraphics[width=2.5in]{DielectricStructure}
%\caption{Layered Dielectric Structure}
%\label{fig:layers}
%\end{figure}

 The Scattering Matrix Method is best utilized when dealing with layered 
 structures with different electrical characteristics, such as the general form 
 in Fig. 2.%\ref{fig:layers}.  
 Each layer would have its own scattering matrix, 
 which would then be combined with others to form a global matrix for the 
 structure.  However, in the normal derivations of the scattering matrices, each 
 matrix depends on the material on either side.  In order to interchange each 
 matrix in a periodic structure, and parallelize the computation, each layer can 
 be modeled as having free space on either side.  The layers will have zero 
 thickness, therefore not affecting the overall solution, but will ease the 
 computation of each scattering matrix.   

%\begin{figure}[here]
%\centering
%\includegraphics[width=2.5in]{ScatterMatrixDiagram}
%\caption{Scatter Matrix Diagram}
%\label{fig:ScatterDiagram}
%\end{figure}

 The scattering matrix, seen in Fig. 2. %\ref{fig:ScatterDiagram}
 , for each layer 
 will relate $A^+$ to $A^-$ from (\ref{eq:10}) and (\ref{eq:11}), equivalently 
 the incident and reflected waves.  When combined with all other layers, the 
 global scattering matrix will give a relation between the total incident wave 
 and the reflected and transmitted waves.  The eigensolutions provide the 
 building blocks for creating the scattering matrix, if we arrange them as 
 follows: 

\begin{equation}
  A_i = W_i^{-1}W_0 + V_i^{-1}V_0
  \label{eq:12}
  \end{equation}
\begin{equation}
  B_i = W_i^{-1}W_0 - V_i^{-1}V_0
  \label{eq:13}
  \end{equation}
\begin{equation}
  X_i = e^{-\lambda_i k_0 L_i}
  \label{eq:14}
  \end{equation}

The A and B parameters are measures of how much each layer’s eigenvector 
deviates from the boundary, in this case free space.  X is a measure of the 
propagation constant, scaled by the wavenumber and the length of each individual 
layer.  If the material is assumed to be lossless, there is a symmetry in the 
scattering matrix, and only two values need to be calculated.  The equations for 
$S_{11}$ and $S_{12}$ are given here, and are equal to $S_{22}$ and $S_{21}$ 
respectively:

\begin{equation}
  S_{11} = (A - BXA^{-1}XB)^{-1}(XBA^{-1}XA - B)
  \label{eq:15}
\end{equation}
\begin{equation}
  S_{12} = (A - BXA^{-1}XB)^{-1}X(A - BA^{-1}B)
  \label{eq:16}
  \end{equation}


